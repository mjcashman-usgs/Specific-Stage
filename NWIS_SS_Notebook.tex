\documentclass[]{article}
\usepackage{lmodern}
\usepackage{amssymb,amsmath}
\usepackage{ifxetex,ifluatex}
\usepackage{fixltx2e} % provides \textsubscript
\ifnum 0\ifxetex 1\fi\ifluatex 1\fi=0 % if pdftex
  \usepackage[T1]{fontenc}
  \usepackage[utf8]{inputenc}
\else % if luatex or xelatex
  \ifxetex
    \usepackage{mathspec}
  \else
    \usepackage{fontspec}
  \fi
  \defaultfontfeatures{Ligatures=TeX,Scale=MatchLowercase}
\fi
% use upquote if available, for straight quotes in verbatim environments
\IfFileExists{upquote.sty}{\usepackage{upquote}}{}
% use microtype if available
\IfFileExists{microtype.sty}{%
\usepackage{microtype}
\UseMicrotypeSet[protrusion]{basicmath} % disable protrusion for tt fonts
}{}
\usepackage[margin=1in]{geometry}
\usepackage{hyperref}
\hypersetup{unicode=true,
            pdftitle={Specific Stage Analysis},
            pdfauthor={Matthew J Cashman},
            pdfborder={0 0 0},
            breaklinks=true}
\urlstyle{same}  % don't use monospace font for urls
\usepackage{color}
\usepackage{fancyvrb}
\newcommand{\VerbBar}{|}
\newcommand{\VERB}{\Verb[commandchars=\\\{\}]}
\DefineVerbatimEnvironment{Highlighting}{Verbatim}{commandchars=\\\{\}}
% Add ',fontsize=\small' for more characters per line
\usepackage{framed}
\definecolor{shadecolor}{RGB}{248,248,248}
\newenvironment{Shaded}{\begin{snugshade}}{\end{snugshade}}
\newcommand{\KeywordTok}[1]{\textcolor[rgb]{0.13,0.29,0.53}{\textbf{#1}}}
\newcommand{\DataTypeTok}[1]{\textcolor[rgb]{0.13,0.29,0.53}{#1}}
\newcommand{\DecValTok}[1]{\textcolor[rgb]{0.00,0.00,0.81}{#1}}
\newcommand{\BaseNTok}[1]{\textcolor[rgb]{0.00,0.00,0.81}{#1}}
\newcommand{\FloatTok}[1]{\textcolor[rgb]{0.00,0.00,0.81}{#1}}
\newcommand{\ConstantTok}[1]{\textcolor[rgb]{0.00,0.00,0.00}{#1}}
\newcommand{\CharTok}[1]{\textcolor[rgb]{0.31,0.60,0.02}{#1}}
\newcommand{\SpecialCharTok}[1]{\textcolor[rgb]{0.00,0.00,0.00}{#1}}
\newcommand{\StringTok}[1]{\textcolor[rgb]{0.31,0.60,0.02}{#1}}
\newcommand{\VerbatimStringTok}[1]{\textcolor[rgb]{0.31,0.60,0.02}{#1}}
\newcommand{\SpecialStringTok}[1]{\textcolor[rgb]{0.31,0.60,0.02}{#1}}
\newcommand{\ImportTok}[1]{#1}
\newcommand{\CommentTok}[1]{\textcolor[rgb]{0.56,0.35,0.01}{\textit{#1}}}
\newcommand{\DocumentationTok}[1]{\textcolor[rgb]{0.56,0.35,0.01}{\textbf{\textit{#1}}}}
\newcommand{\AnnotationTok}[1]{\textcolor[rgb]{0.56,0.35,0.01}{\textbf{\textit{#1}}}}
\newcommand{\CommentVarTok}[1]{\textcolor[rgb]{0.56,0.35,0.01}{\textbf{\textit{#1}}}}
\newcommand{\OtherTok}[1]{\textcolor[rgb]{0.56,0.35,0.01}{#1}}
\newcommand{\FunctionTok}[1]{\textcolor[rgb]{0.00,0.00,0.00}{#1}}
\newcommand{\VariableTok}[1]{\textcolor[rgb]{0.00,0.00,0.00}{#1}}
\newcommand{\ControlFlowTok}[1]{\textcolor[rgb]{0.13,0.29,0.53}{\textbf{#1}}}
\newcommand{\OperatorTok}[1]{\textcolor[rgb]{0.81,0.36,0.00}{\textbf{#1}}}
\newcommand{\BuiltInTok}[1]{#1}
\newcommand{\ExtensionTok}[1]{#1}
\newcommand{\PreprocessorTok}[1]{\textcolor[rgb]{0.56,0.35,0.01}{\textit{#1}}}
\newcommand{\AttributeTok}[1]{\textcolor[rgb]{0.77,0.63,0.00}{#1}}
\newcommand{\RegionMarkerTok}[1]{#1}
\newcommand{\InformationTok}[1]{\textcolor[rgb]{0.56,0.35,0.01}{\textbf{\textit{#1}}}}
\newcommand{\WarningTok}[1]{\textcolor[rgb]{0.56,0.35,0.01}{\textbf{\textit{#1}}}}
\newcommand{\AlertTok}[1]{\textcolor[rgb]{0.94,0.16,0.16}{#1}}
\newcommand{\ErrorTok}[1]{\textcolor[rgb]{0.64,0.00,0.00}{\textbf{#1}}}
\newcommand{\NormalTok}[1]{#1}
\usepackage{graphicx,grffile}
\makeatletter
\def\maxwidth{\ifdim\Gin@nat@width>\linewidth\linewidth\else\Gin@nat@width\fi}
\def\maxheight{\ifdim\Gin@nat@height>\textheight\textheight\else\Gin@nat@height\fi}
\makeatother
% Scale images if necessary, so that they will not overflow the page
% margins by default, and it is still possible to overwrite the defaults
% using explicit options in \includegraphics[width, height, ...]{}
\setkeys{Gin}{width=\maxwidth,height=\maxheight,keepaspectratio}
\IfFileExists{parskip.sty}{%
\usepackage{parskip}
}{% else
\setlength{\parindent}{0pt}
\setlength{\parskip}{6pt plus 2pt minus 1pt}
}
\setlength{\emergencystretch}{3em}  % prevent overfull lines
\providecommand{\tightlist}{%
  \setlength{\itemsep}{0pt}\setlength{\parskip}{0pt}}
\setcounter{secnumdepth}{0}
% Redefines (sub)paragraphs to behave more like sections
\ifx\paragraph\undefined\else
\let\oldparagraph\paragraph
\renewcommand{\paragraph}[1]{\oldparagraph{#1}\mbox{}}
\fi
\ifx\subparagraph\undefined\else
\let\oldsubparagraph\subparagraph
\renewcommand{\subparagraph}[1]{\oldsubparagraph{#1}\mbox{}}
\fi

%%% Use protect on footnotes to avoid problems with footnotes in titles
\let\rmarkdownfootnote\footnote%
\def\footnote{\protect\rmarkdownfootnote}

%%% Change title format to be more compact
\usepackage{titling}

% Create subtitle command for use in maketitle
\newcommand{\subtitle}[1]{
  \posttitle{
    \begin{center}\large#1\end{center}
    }
}

\setlength{\droptitle}{-2em}

  \title{Specific Stage Analysis}
    \pretitle{\vspace{\droptitle}\centering\huge}
  \posttitle{\par}
    \author{Matthew J Cashman}
    \preauthor{\centering\large\emph}
  \postauthor{\par}
      \predate{\centering\large\emph}
  \postdate{\par}
    \date{September 30, 2018}


\begin{document}
\maketitle

\section{Pre-analysis}\label{pre-analysis}

\subsection{Parameter selection}\label{parameter-selection}

Before we begin with our specific stage analysis, there are several
settings that we will have to define first. 1. Site - USGS site number
to do the analysis on. 2. startDate - Beginning of the period of
interest, if available. Blank will pull from the earliest record. 3.
endDate - End of the period of interest. Blank will pull from the latest
record. 4. Quantiles - Mean daily flow quantiles, in decimal form, for
the analysis. e.g.~0.8 is the 80\% of mean daily flow, or 20\%
exceedance.5. Sensitivity - Used in the clean up procedure. Will reject
any values outside of discharge quantile equation: +/- sensitivity, in
percent 6. target - Target quantile of interest, used to create single,
focused output graphs

\begin{Shaded}
\begin{Highlighting}[]
\NormalTok{Site <-}\StringTok{ "01589035"}
\NormalTok{startDate <-}\StringTok{ ""}
\NormalTok{endDate <-}\StringTok{ ""}
\NormalTok{Quantiles <-}\StringTok{ }\KeywordTok{c}\NormalTok{(}\FloatTok{0.3}\NormalTok{,}\FloatTok{0.5}\NormalTok{,}\FloatTok{0.7}\NormalTok{,}\FloatTok{0.8}\NormalTok{,}\FloatTok{0.85}\NormalTok{,}\FloatTok{0.9}\NormalTok{,}\FloatTok{0.95}\NormalTok{,}\FloatTok{0.98}\NormalTok{,}\FloatTok{0.99}\NormalTok{,}\FloatTok{0.9975}\NormalTok{,}\FloatTok{0.999}\NormalTok{,}\FloatTok{0.9999}\NormalTok{,}\FloatTok{0.9999999}\NormalTok{)}
\NormalTok{Sensitivity <-}\StringTok{ }\FloatTok{0.01}
\NormalTok{target <-}\StringTok{ }\DecValTok{8}
\end{Highlighting}
\end{Shaded}

\subsection{Load NWIS data}\label{load-nwis-data}

Here we are loading data that has been loaded locally. Options exist to
pull directly from NWIS using the dataRetrieval package. Error values or
missing data (e.g.~values of -999999) are excluded at this point.

\begin{verbatim}
## Reading Daily Data locally
\end{verbatim}

\begin{verbatim}
## Reading Continuous Data locally
\end{verbatim}

\begin{verbatim}
## Reading Site Info locally
\end{verbatim}

\subsection{Plot NWIS data}\label{plot-nwis-data}

After our data has been downloaded, we will plot mean daily discharge,
instantaneous unit discharge, and instantaneous unit stage to check for
any gaps or weird values. Directly measured values are indicated by red
points.

\includegraphics{NWIS_SS_Notebook_files/figure-latex/Plot Loaded DV/UV Data-1.pdf}
\includegraphics{NWIS_SS_Notebook_files/figure-latex/Plot Loaded DV/UV Data-2.pdf}
\includegraphics{NWIS_SS_Notebook_files/figure-latex/Plot Loaded DV/UV Data-3.pdf}

\section{Specific Stage Analysis}\label{specific-stage-analysis}

\subsection{Run the Analysis}\label{run-the-analysis}

\begin{Shaded}
\begin{Highlighting}[]
\CommentTok{#Run Specific Stage Analysis ----}
\NormalTok{SS_results <-}\StringTok{ }\KeywordTok{SS}\NormalTok{(Site, Quantiles)}
\end{Highlighting}
\end{Shaded}

\subsection{Cleanup}\label{cleanup}

Now we check results for stage and discharge to make sure there are no
outliers. In the figure below, we can see some scatter around our target
discharge quantiles.

\includegraphics{NWIS_SS_Notebook_files/figure-latex/Plot Check for Outliers-1.pdf}

A simple dplyr pipe uses mutate to calculate the percent difference
between the calculated discharge and the discharge of interest. Results
are then filtered to select only those values less than the percent
difference specified by Sensitivity.

\begin{Shaded}
\begin{Highlighting}[]
\NormalTok{SS_results_clean <-}\StringTok{ }\NormalTok{SS_results }\OperatorTok
\StringTok{                    }\KeywordTok{mutate}\NormalTok{(}\DataTypeTok{pc_diff =}\NormalTok{ (Flow_Inst}\OperatorTok{-}\NormalTok{Quantflow)}\OperatorTok{/}\NormalTok{Quantflow) }\OperatorTok
\StringTok{                    }\KeywordTok{filter}\NormalTok{(pc_diff }\OperatorTok{>}\StringTok{ }\OperatorTok{-}\NormalTok{Sensitivity }\OperatorTok{&}\StringTok{ }\NormalTok{pc_diff }\OperatorTok{<}\StringTok{ }\NormalTok{Sensitivity)}
\end{Highlighting}
\end{Shaded}

\subsection{Post-outlier Check}\label{post-outlier-check}

Now we check results again to make sure all outliers have been removed.
\includegraphics{NWIS_SS_Notebook_files/figure-latex/Plot Cleaned results-1.pdf}

\section{Specific Stage Result Plots}\label{specific-stage-result-plots}

Now that outliers have been removed, we can print all our results. First
in one plot for all quantiles. Next are individual plots for all target
quantiles

\subsection{All Quantiles Plot}\label{all-quantiles-plot}

\includegraphics{NWIS_SS_Notebook_files/figure-latex/unnamed-chunk-2-1.pdf}

\subsection{Target Quantile Plots}\label{target-quantile-plots}

\includegraphics{NWIS_SS_Notebook_files/figure-latex/unnamed-chunk-3-1.pdf}
\includegraphics{NWIS_SS_Notebook_files/figure-latex/unnamed-chunk-3-2.pdf}
\includegraphics{NWIS_SS_Notebook_files/figure-latex/unnamed-chunk-3-3.pdf}
\includegraphics{NWIS_SS_Notebook_files/figure-latex/unnamed-chunk-3-4.pdf}
\includegraphics{NWIS_SS_Notebook_files/figure-latex/unnamed-chunk-3-5.pdf}
\includegraphics{NWIS_SS_Notebook_files/figure-latex/unnamed-chunk-3-6.pdf}
\includegraphics{NWIS_SS_Notebook_files/figure-latex/unnamed-chunk-3-7.pdf}
\includegraphics{NWIS_SS_Notebook_files/figure-latex/unnamed-chunk-3-8.pdf}
\includegraphics{NWIS_SS_Notebook_files/figure-latex/unnamed-chunk-3-9.pdf}
\includegraphics{NWIS_SS_Notebook_files/figure-latex/unnamed-chunk-3-10.pdf}
\includegraphics{NWIS_SS_Notebook_files/figure-latex/unnamed-chunk-3-11.pdf}
\includegraphics{NWIS_SS_Notebook_files/figure-latex/unnamed-chunk-3-12.pdf}
\includegraphics{NWIS_SS_Notebook_files/figure-latex/unnamed-chunk-3-13.pdf}

\section{Let's explore site
measurements}\label{lets-explore-site-measurements}

\subsection{Download site measurement
data}\label{download-site-measurement-data}

\begin{Shaded}
\begin{Highlighting}[]
\NormalTok{rating<-}\KeywordTok{readNWISrating}\NormalTok{(}\DataTypeTok{siteNumber=}\NormalTok{ Site, }\StringTok{"exsa"}\NormalTok{)}
\KeywordTok{head}\NormalTok{(rating)}
\end{Highlighting}
\end{Shaded}

\begin{verbatim}
##   INDEP SHIFT  DEP STOR
## 1  4.41 -0.01 5.00 <NA>
## 2  4.42 -0.01 5.28 <NA>
## 3  4.43 -0.01 5.58 <NA>
## 4  4.44 -0.01 5.88 <NA>
## 5  4.45 -0.01 6.19 <NA>
## 6  4.46 -0.01 6.51 <NA>
\end{verbatim}

\begin{Shaded}
\begin{Highlighting}[]
\KeywordTok{tail}\NormalTok{(rating)}
\end{Highlighting}
\end{Shaded}

\begin{verbatim}
##      INDEP SHIFT      DEP STOR
## 2075 25.15     0 27250.32 <NA>
## 2076 25.16     0 27260.25 <NA>
## 2077 25.17     0 27270.19 <NA>
## 2078 25.18     0 27280.13 <NA>
## 2079 25.19     0 27290.07 <NA>
## 2080 25.20     0 27300.00    *
\end{verbatim}


\end{document}
